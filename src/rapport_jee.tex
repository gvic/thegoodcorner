\documentclass[]{report}
\usepackage[T1]{fontenc}
\usepackage[graphicx]{graphicx}
\usepackage[subfigure]{subfigure}
\usepackage[french]{babel}
\usepackage[latin1]{inputenc}

\title{Rapport du projet JEE}

\author{Bonaud Paul\\
	Garcez Maxime\\
	Godayer Victor\\
	Grihangne Martin}

\date{}\today

\begin{document}

\section{Pr�sentation du site}

\subsection{Concept}
Le site web que nous avons d�velopp� s'inspire d'un concept classique des sites de e-commerce, comme le site bien connu "leboncoin.com"
L'application d�velopp�e permet � un internaute de poster et de consulter des annonces de vente de biens divers.
En plus des fonctionnalit�s typiques de ce genre de site, le site offre la possibilit� in�dite de cr�er des communaut�s d'utilisateurs partageant
le m�me centre d'int�ret ou de appartenant � un m�me groupe(professionnel, culturel, g�ographique, etc.).
Par exemple, une �cole peut utiliser cette fonctionnalit� pour permettre � ses �l�ves de revendre leurs livres scolaires d'une ann�e � l'autre. 
De m�me l'ENSEEIHT pourrait avoir sa propre communaut� afin de faciliter les �changes et ventes de biens entres ses �l�ves.



\section{Architecture }
\subsection{Mod�le des donn�es}


\section{Technologies}
\subsection{}

\section{Bilan}
\subsection{}

\end{document}
